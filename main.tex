% 
% ======================================================================
\RequirePackage{docswitch}
% \flag is set by the user, through the makefile:
%    make note
%    make apj
% etc.
\setjournal{\flag}

\documentclass[\docopts]{\docclass}

% You could also define the document class directly
%\documentclass[]{emulateapj}

% Custom commands from LSST DESC, see texmf/styles/lsstdesc_macros.sty
\usepackage{lsstdesc_macros}

\usepackage{graphicx}
\graphicspath{{./}{./figures/}}
\bibliographystyle{apj}
\usepackage{subfigure}

\usepackage{draftwatermark}
\SetWatermarkScale{1}
\SetWatermarkLightness{0.90}

% Add your own macros here:



% 
% ======================================================================

\begin{document}

\title{ On the cadence of the LSST SN survey(s) }

\maketitlepre

\begin{abstract}

  % We discuss key design elements that must be taken into account to
  % maximize the impact of the LSST deep and wide SN surveys. We show
  % that they can be compressed into simple requirements on the quality
  % of the SN light curves.
  
  We describe a simple metric that allows to evaluate the quality of
  the SN~Ia light curves that will be delivered by the LSST Wide and
  DDF surveys.  Evaluating this metric does not require to generate
  and fit SN lightcurves.  It can therefore be used to evaluate
  efficiently years of survey operations. It simple enough to be
  included in the survey operation scheduler.
  
  Using this approach, we evaluate a small subset of the \code{OpSim}
  \code{Minion\_1016} cadence. We find that the cadence and depth
  described in \code{Minion\_1016} do not allow to build a redshift
  limited survey up to $z \sim 1$.

  The results presented in this work depend heavily on our knowledge
  of (1) the throughput of the LSST telescope and camera system and
  (2) the median observing conditions at Cerro Pachon (sky brightness
  and seeing).  We compare the two documented LSST instrument models
  (\code{LSE-40} and \code{SMTN-002}) with \code{Minion\_1016}.  The
  5-$\sigma$ depth obtained with \code{Minion\_1016} is about 0.8-mag
  lower than what could be inferred a few years ago using the
  \code{LSE-40} instrument model (and the median seeing and sky
  brightnes values reported therein).  This has a very significant
  impact on the effective depth of a LSST SN survey.
  
  The results presented in this note will be updated as new
  \code{OpSim} realizations become available.
\end{abstract}

% Keywords are ignored in the LSST DESC Note style:
\dockeys{latex: templates, papers: awesome}

\maketitlepost

% ----------------------------------------------------------------------
% 

\section{Introduction}
\label{sec:intro}


In this note, we examine the cadence envisioned for the 10 years of
LSST operations.  In \S\ref{sec:design_notes} we discuss the key
requirements that drive the design of a SN survey.  We then
(\S\ref{sec:metric}) present a very simple metric that will allow us
to evaluate effectively a survey cadence without having to rely on
extensive simulations.  In the two next sections
(\S\ref{sec:ddf_cadence} and \S\ref{sec:wide_cadence}), we apply this
metric to the \code{OpSim} DDF and wide cadence respectiveley. We
conclude in \ref{sec:conclusions}.

% We base ourselves on the \code{Minion\_1016} run of \code{OpSim}. The
% results presented here will be updated as new \code{OpSim} runs become
% available.





% ----------------------------------------------------------------------

\section{Design Elements}
\label{sec:design_notes}

With an adapted rolling cadence, LSST has the capability to discover
and follow-up  $10^4$ to $10^5$ SNe~Ia in the redshift range $0.1 < z <
1$ (Wide and DDF surveys combined) and to build a redshift-limited
sample up to $z \sim 1$. The distant part of this Hubble diagram will
be competitive with the DESI constraints. At low redshifts ($z < 0.5$)
LSST will have essentially no competitor.

Today's SN~Ia Hubble diagrams are systematics dominated.  When
designing a future high-statistics SN survey, provisions and plans
must be made to push down as much as possible the level of the (known)
sources of systematics. Photometric calibration is today the dominant
contribution to the systematic error budget.  There is ongoing work in
the Project and in DESC to lower its contribution by a factor $\sim 5$
w.r.t.  today's standards.

One important systematics that can be eliminated entirely at the
survey design stage is the Malmquist bias.  It impacts the faint end
of the wide and DDF surveys ($z \sim 0.4$ and $z \sim 0.9$ resp.)
where one has to model the fraction of events lost.  This procedure
yields uncertainties, that are quite intricated with (1) the control
of the demographic evolution and the intrinsic properties of SNe~Ia
(2) the standardization procedures (see impact on $\beta$).  As a
consequence, the supernovae beyond $z_{lim}$ are of limited
usefulness.  This limits the actual depth and therefore the
cosmological impact of the survey. The key is therefore to select a
cadence that allows one to build a {\em redshit limited sample}, up to
a certain redshift limit, $z_{lim}$. A sound approach, is to set a
redshift limit (this is a scientific decision, based on what we expect
from the SN~Ia Hubble diagram in the global cosmology fit), and then
to design the cadence that permit to build a complete sample up to
that redshift limit.

\paragraph{Redshift-limited sample} Note that this ``redshift-limit''
is not a detectability limit. It is the redshift value beyond which
we start losing a fraction of the events, because their photometric
follow-up is not good enough to (1) measure a distance and (2) perform
a photometric identification.

% Rather, we need to define, for each of our two surveys (DDF and wide)
% a nominal $z_{lim}$, and aim at building a sample that is complete up
% to $z_{lim}$.

A redshift limited sample is a sample such that every SN~Ia that occurs:
(1) in a well defined observer-frame time interval (that corresponds
  to a $\sim 180$ day search season) $\mathrm{[MJD_{start}; MJD_{end}]}$;
(2) in the redshift range $z < z_{lim}$;
(3) in a region of the SN~Ia parameter space that is large enough to
  encompass a potential evolution with redshift of the SN~Ia
  demographic properties;
has a follow-up which is ``good enough'' to (1) identify it
photometrically as a SN~Ia and (2) derive a standardized distance from
its lightcurve.

To define the parameter space region of interest, we propose to
proceed as follows: up to a good approximation, SNe~Ia form a
2-dimensional family, that may be indexed for example with color and
lightcurve-width (for example, the SALT2-color and the
SALT2-$X_1$-parameter).  On figure \ref{fig:jla_X1_C}, we show the
distribution of the JLA supernovae, in the $(X_1,C)$ parameter
space. We note sizeable differences between the nearby and distant
SNe, in particular in $X_1$.  However, we see that the full JLA sample
is comprised in the region $X_1 = [-3,3], Color= = [-0.3, 0.3]$.

The much larger LSST samples will contain events that are outside that
region of interest.  However, we can infer from JLA that the core of
the distribution of SN~Ia is well contained in it -- at all redshifts
below 1.


\begin{figure}[t]
\begin{center}
\includegraphics[width=0.75\linewidth]{sn_parameter_space.pdf}
\caption{JLA supernovae the $(X_1,Color)$ parameter space -- (blue:
  nearby, green: SDSS, orange: SNLS).  }
\label{fig:jla_X1_C}
\end{center}
\end{figure}

On the same figure, we have marked in red the faintest SN in that
parameter space ($X_1=-3, Color=0.3$). To be complete up to $z_{lim}$,
the survey has to deliver a cadence that allows to derive a luminosity
distance and a photometric identification for an event of such $X_1$
and Color, at a redshift $z_{lim}$.

\paragraph{Lightcurve quality requirements} Light curve quality
requirements have been discussed a few years ago, for the LSST-Euclid
paper \citep{2014A&A...572A..80A}. We summarize them below:

\begin{itemize}
\item the follow-up of each supernova must be good enough in the
  observer-frame bands that correspond to the $B$- and $V$-restframe
  spectrum ($3800 \angstrom < \lambda < 7000 \angstrom$).  At
  high-redshift, in particular, one should avoid relying on the $UV$
  restframe region to derive a distance, given the high intrinsic
  dispersion of SN~Ia at those wavelengths.
  
\item we require the light curve shape to be well sampled in the
  (restframe) phase interval $[-10;+30]$ days, with at least four
  visits before peak (each of those visits in any of the eligible
  band), and ten visits after peak.  To obtain this in the lower
  redshift region of the Hubble diagram, one requires an
  observer-frame cadence of 4 days.  In the upper redshift region (DDF
  fields), this requirement may be slightly relaxed. However, since we
  are going to rely exclusively on photometric identification, it is
  essential to secure a tight sampling of the SN color evolution at
  all redshift.
  
\item we require that the photon noise contribution to the distance
  measurement is subdominant w.r.t. the intrinsic dispersion of the
  SNe (after standardization).  There are several ways to quantify
  this.  With today's standardization techniques, the SN standardized
  distance modulus is:
  \begin{equation}
    \mu = m^\star_B + \alpha X_1 - \beta C - \cal{M}
  \end{equation}
  where $m^\star_B$ is the peak brightness in restframe $B$, $X_1$
  characterize the lightcurve width, and $C$ is an estimate of the
  restframe color $B-V$. $\alpha$, $\beta$ and $\cal{M}$ are global
  parameters, fit along with the cosmology. If the light curve is
  correctly sampled (see point above), the propagation of the
  measurement uncertainties affecting $m^\star_B$, $X_1$ and $C$ is
  dominated by the contribution of $\sigma_C$. Indeed, if the light
  curve is correctly sampled
  

  The dominant contribution is carried by the color (since
  $\beta \sim 3$). This means that requiring $\sigma C < 0.03$ ensures
  that $\sigma \mu < 0.1$, below the intrinsic dipersion in the Hubble
  diagram, after standardization.


\item each SN must have good quality measurements in at least three
  bands. First, we need indeed to constrain the restframe color of the
  SN, with two bands covering the restframe $B$ and $V$ region. Then,
  we

  

  Another way to look at this consists in placing requirements on the
  uncertainty of the amplitude of the lightcurve shape. 
\end{itemize}



Our goal is to make sure that the light curves of our faintest
$(X_1=-3, C=0.3)$ SNe~Ia around $z = z_{lim}$ pass these requirements.


% ----------------------------------------------------------------------

\section{A metric to evaluate a cadence}
\label{sec:metric}



\subsection{The signal-to-noise on the light-curve amplitude}

In this section, we present a metric 

If we fit a model $A \times \ell(t)$ on a lightcurve $(t_i, y_i)$, the
least square estimate of the amplitude is:
\begin{equation}
  \hat{A} = \frac{\sum_i w_i \ell_i y_i}{\sum_i w_i \ell^2_i}
\end{equation}
where we note $\ell(t_i) = \ell_i$. The  variance of $\hat{A}$ is:
\begin{equation}
  \mathrm{Var}(\hat{A}) = \left(\sum_i w_i \ell^2_i\right)^{-1}
\end{equation}
and signal to noise we get on the amplitude $\hat{A} /
\sigma_{\hat{A}}$ is:
\begin{equation}
  SNR = \left(\sum_i w_i L^2_i\right)^{1/2}
\end{equation}
where $L_i = A \times \ell_i$. Since we are dominated by the sky
background, the
\begin{equation}
  SNR = 5 \times \left(\sum_i L^2_i f^{-2}_{i|5}\right)^{1/2}
\end{equation}
This metrics is easy to compute, all it takes is a tabulated model of
the light curve shape, and a cadence file, containing, for each visit,
an assessment of the $5\sigma$-depth reached during that visit. 


\subsection{Requirements on the survey cadence and depth}

Using the metric above, we can give characterize the cadence and depth
that allow to fulfill the requirements above. 

Let's define $\delta(t_i) = \delta_i$, which is equal to the number of
observations in an interval $\Delta t$ around $t_i$. The metric above
can be rewritten:
\begin{equation}
  SNR = 5 \times \left(\sum_i \delta_i L^2_i f^{-2}_{i|5} \Delta t\right)^{1/2}
\end{equation}
where the sum runs on all the $\Delta t$ bins, in a observer frame
time interval covering the supernova lightcurve evolution. The requirement 
above can be rewritten:
\begin{equation}
  f_{|5} \left<\delta_i\right>^{-1/2} \leq \frac{5 \sqrt{\Delta t} \sqrt{\sum_i L_i^2}}{SNR}
  \label{eqn:global_metric}
\end{equation}
where $\left<\delta_i\right>$ is the cadence, weighted by the light
curve shape squared: $\left<\delta_i\right> = \sum_i \delta_i
L_i^2/\sum_i L_i^2$.  This sets a limit (in $\frac{e^-}{s}
\sqrt{day}$) on the product of the limiting flux times inverse square
root of the cadence.

This simple metric allows one to estimate the power of a cadence
without having to generate and fit supernova light curves.  Of course,
a supernova scientist is still needed, to compute the numerical values
of the upper limits that appears in equation \ref{eqn:global_metric}
For the records, we report these values in table
\ref{tab:cadence_depth_limit}. They have been computed with SALT-2.4
and the SMTN-002 instrument model.



\begin{table}[t]
\begin{center}
\caption{cadence-depth limit for the worst case SN and a target SNR=20/band}
\label{tab:cadence_depth_limit}
\begin{tabular}{l|rrrrr}
\hline
\hline
    $z$   &      $g$         &       $r$         &     $i$           &      $z$        &      $y$           \\
          &      \multicolumn{5}{c}{$[\mathrm{e^-/s \times \sqrt{day}}]$} \\
\hline
     0.2  &     916 &    1330 &    1262 &     704 &     289   \\
     0.3  &     284 &     513 &     495 &     421 &     149   \\
     0.4  &      86 &     267 &     279 &     224 &      91   \\
     0.5  &      31 &     153 &     148 &     131 &      76   \\
     0.6  &      12 &      81 &      94 &      92 &      43   \\
     0.7  &         &      42 &      69 &      57 &      34   \\
     0.8  &         &      17 &      52 &      39 &      25   \\
     0.9  &         &      11 &      32 &      31 &      17   \\
     1.0  &         &         &      20 &      26 &      12   \\
     1.1  &         &         &      11 &      20 &      11   \\

     % 0.4  &      86 &     267 &     279 &         &           \\
     % 0.5  &      31 &     153 &     148 &         &           \\
     % 0.6  &      12 &      81 &      94 &      92 &           \\
     % 0.7  &         &      42 &      69 &      57 &      34   \\
     % 0.8  &         &      17 &      52 &      39 &      25   \\
     % 0.9  &         &      11 &      32 &      31 &      17   \\
     % 1.1  &         &         &      11 &      20 &      11   \\
\hline
\end{tabular}
\end{center}
\end{table}



% ----------------------------------------------------------------------
\section{The DEEP LSST SN survey}
\label{sec:ddf_cadence}
We now apply these two metrics 

\subsection{The DDF cadence from \code{Minion\_1016}}
\label{sec:results}




\begin{figure*}
\begin{center}
\subfigure[$r$]{\includegraphics[width=0.48\linewidth]{m5_cadence_limits_r.pdf}}
\subfigure[$i$]{\includegraphics[width=0.48\linewidth]{m5_cadence_limits_i.pdf}}\\
\subfigure[$z$]{\includegraphics[width=0.48\linewidth]{m5_cadence_limits_z.pdf}}
\subfigure[$y$]{\includegraphics[width=0.48\linewidth]{m5_cadence_limits_y.pdf}}
\caption{The cadence {\em vs.} depth requirements for the DDF LSST SN
  survey in the $r,i,z$ and $y$-bands. The color scale corresponds to
  the metric described in equation \ref{eqn:global_metric}.  The lines are
  the contour-levels computed for the limits indicated on table
  \ref{tab:cadence_depth_limit}. To fulfill the SNR requirements at a
  given redshift, one has to be {\em below} the corresponding
  line. The stars indicate an ambitious -- but attainable -- goal for
  a DDF survey.  The red crosses show what the survey can currently
  deliver according to \code{Minion\_1016}. }
\label{fig:m5_cadence_limits_ddf}
\end{center}
\end{figure*}



\begin{figure*}[t]
  \begin{center}
    \includegraphics[width=\linewidth]{metric_DD_290.pdf}
    \caption{}
  \end{center}
\end{figure*}




% ----------------------------------------------------------------------
\section{The Wide LSST SN survey}
\label{sec:wide_cadence}

\subsection{The cadence from \code{Minion\_1016}}
\label{sec:results}

\begin{figure*}
\begin{center}
\subfigure[$g$]{\includegraphics[width=0.48\linewidth]{m5_cadence_limits_wide_g.pdf}}
\subfigure[$r$]{\includegraphics[width=0.48\linewidth]{m5_cadence_limits_wide_r.pdf}}\\
\subfigure[$i$]{\includegraphics[width=0.48\linewidth]{m5_cadence_limits_wide_i.pdf}}
\subfigure[$z$]{\includegraphics[width=0.48\linewidth]{m5_cadence_limits_wide_z.pdf}}
\caption{The cadence {\em vs.} depth requirements for the Wide survey,
  in the $g, r, i$ and $z$-bands. The color scale corresponds to the
  metric described in equation \ref{eqn:global_metric}.  The lines are
  the contour-levels computed for the limits indicated on table
  \ref{tab:cadence_depth_limit}. To fulfill the SNR requirements at a
  given redshift, one has to be {\em below} the corresponding
  line. The stars indicate an ambitious -- but attainable -- goal for
  a DDF survey.  The red crosses show what the survey can currently
  deliver according to \code{Minion\_1016}.}
\label{fig:m5_cadence_limits_wide}
\end{center}
\end{figure*}

\begin{figure*}[t]
  \begin{center}
    \includegraphics[width=\linewidth]{metric_WFD_309.pdf}
    \caption{}
  \end{center}
\end{figure*}



% \figref{example} shows an example figure, referred to with the \verb=\figref= command and the \code{fig:} prefix.
% \begin{figure}
% \includegraphics[width=0.9\columnwidth]{example.png}
% \caption{An example figure: the LSST DESC logo, copied from \code{texmf/logos/desc-logo.png} into \code{figures/example.png}. \label{fig:example}}
% \end{figure}

\subsection{The rolling cadence from \code{Minion\_1016}}
\label{sec:results}

% ----------------------------------------------------------------------

\section{The Rolling Cadence }
\label{sec:results}

% \figref{example} shows an example figure, referred to with the \verb=\figref= command and the \code{fig:} prefix.

% \begin{figure}
% \includegraphics[width=0.9\columnwidth]{example.png}
% \caption{An example figure: the LSST DESC logo, copied from \code{texmf/logos/desc-logo.png} into \code{figures/example.png}. \label{fig:example}}
% \end{figure}




% ----------------------------------------------------------------------

\section{Discussion}
\label{sec:discussion}

If you are planning on committing your paper to GitHub, it's a good idea to write your tex as one sentence per line.
This allows for an easier \code{diff} of changes.
It also makes sense to think of latex as \emph{code}, and sentences as logical statements, occupying one line each.
Each line must ``compile'' in the mind of the reader.


% ----------------------------------------------------------------------

\section{Conclusions}
\label{sec:conclusions}

We have presented design guidelines for the wide and deep LSST SN
surveys and discussed requirements on the survey depth and cadence.
We have developed a lightweight metric to assess whether a given
cadence matches these requirements without having to resort to an
extensive simulation.

We have evaluated the \code{Minion\_1016} cadence, on a series of DDF
and wide fields. We summarize our conclusions below:

\begin{itemize}
\item on the DDF fields, the survey is complete up to $z \sim
  0.65$. This is comparable to what was obtained with SNLS in the
  2000's but well below what should be an ambitious goal for LSST
  (produce a redshift limited SN~Ia Hubble diagram up to $z_{lim} \sim
  1$).

\item on the DDF fields, increasing the duration of the search seasons
  from 4.5 months to 6 months would allow to gain tremendously in
  statistics (almost a factor 2) at high redshift.
  
\item on the Wide fields, the redshift limit is of about $z \sim
  0.3$. A more ambitious target is rather $z_{lim} \sim 0.4$.  In the
  $g$-band, the regularity of the cadence should be improved.

\item we have shown that, by tweaking the cadence, one can reach a
  $z_{lim}$ of 0.9 for the DDF fields and $z_{lim} \sim 0.4$ on the
  Wide, in the same observing time budget.

\item it remains to be proven that these cadences can be reached
  within the constraints of the survey operations, as they are
  implemented in {\tt OpSim}.  We suggest implementing, in the OpSim
  scheduler, a dynamic scheduling algorithm based on the metrics
  discussed above.  
  
  To give orders of magnitudes, a reasonable target is a cadence of 4
  restframe days at the median redshift of the wide and deep surveys.
  This corresponds to an observer frame cadence of $\sim 5.2$ (resp
  6.8) days on the wide and the DDF respectively.

\item finally, the results presented here should be taken with a bit
  of caution.  As we discuss in appendix
  \ref{sec:lsst_instrument_models}, and shown on figure
  \ref{fig:lsst_model_summary}, the assessment of the instrument
  throughput and of the average observing conditions have been
  recently revised, and this revision has a very significant impact on
  the expected survey depth (almost half a magnitude per standard
  visit).  

  Furthermore, the average observing conditions used in
  \code{Minion\_{1016}} seem to be quite pessimistic.  In particular,
  the \code{Minion\_{1016}} sky brightness is probably too bright,
  especially in the $y$-band.
\end{itemize}

The results presented in this note will be updated as revised OpSim
cadences become available.


% ----------------------------------------------------------------------

\subsection*{Acknowledgments}

Here is where you should add your specific acknowledgments, remembering that some standard thanks will be added via the \code{acknowledgments.tex} and \code{contributions.tex} files.

\input{acknowledgments}

\input{contributions}

%{\it Facilities:} \facility{LSST}

% Include both collaboration papers and external citations:
\bibliography{lsstdesc,main}





\appendix

\section{LSST instrument models}
\label{sec:lsst_instrument_models}

To prepare this study, we have used two instrument models.  One is
based on the numbers reported in \cite[][LSE-40 hereafter]{LSE-40}.
We report the main ingredients of this model in table \ref{tab:lse40}. 

The other model is described in \cite[][hereafter
SMTN-002)]{SMTN-002}, which constitutes a preliminary update of
LSE-40.  The current version OpSim relies on SMTN-002. An we therefore
adopt this model as our reference. Key quantities of SMTN-002 are
listed in table \ref{tab:smtn002}.

We note that both models differ very significantly. In particular (1)
the throughput of SMTN-002 is almost 50\% lower.


\begin{table}
\begin{center}
\caption{LSE-40 model}
\label{tab:lse40}
\begin{tabular}{l|cccccc}
\hline 
\hline 
\multicolumn{7}{c}{{\bf General}} \\
\hline
Pixel size & \multicolumn{6}{r}{0.2 arcsec} \\
RO noise   & \multicolumn{6}{r}{9 $e^-$}    \\
\hline
\multicolumn{7}{c}{{\bf Zero Points @ X=1 [AB, fluxes in e$^-$/s]}} \\
\hline
           &  $u$ & $g$ & $r$ & $i$ & $z$ & $y$ \\
LSE-40     & 27.09 & 28.58 & 28.50 & 28.34 & 27.95 & 27.18 \\
snsim      & 27.05 & 28.59 & 28.53 & 28.38 & 27.99 & 27.22 \\
\hline
\multicolumn{7}{c}{{\bf median seeing [arcsec]}} \\
\hline
LSE-40 / snsim  &  0.77 &  0.73 &  0.70 &  0.67 &  0.65 &  0.63 \\
\hline
\multicolumn{7}{c}{{\bf Dark sky [AB mag / arcsec$^2$]}}   \\
\hline
LSE-40     & 22.92 & 22.27 & 21.20 & 20.47 & 19.59 & 18.63 \\
snsim      & 22.95 & 22.26 & 21.20 & 20.47 & 19.60 & 18.61 \\
\hline
\multicolumn{7}{c}{{\bf NEA [pixel$^2$]}}   \\
\hline
snsim (Moffat, $\beta=4.5$)     & 41.5  & 37.4  & 34.5  & 31.7 & 29.9  & 28.6  \\
\hline
\multicolumn{7}{c}{{\bf Limiting mag ($5 \sigma$), 30-s visit}}   \\
\hline
LSE-40                        & 24.22  &  25.15 &  24.74  &  24.38  &  23.80  &  22.93  \\
snsim (Moffat, $\beta=7$)     & 24.27  &  25.18 &  24.73  &  24.36  &  23.77  &  22.92  \\
\hline
\end{tabular}
\end{center}
\end{table}


\begin{table}
\begin{center}
\caption{SMTN-002 model}
\label{tab:smtn002}
\begin{tabular}{l|cccccc}
\hline 
\hline 
\multicolumn{7}{c}{{\bf General}} \\
\hline
Pixel size & \multicolumn{6}{r}{0.2 arcsec} \\
RO noise   & \multicolumn{6}{r}{9 $e^-$}    \\
\hline
\multicolumn{7}{c}{{\bf Zero Points @ X=1 [AB, fluxes in e$^-$/s]}} \\
\hline
           &  $u$ & $g$ & $r$ & $i$ & $z$ & $y$ \\
SMTN-002   & 26.50 & 28.30 & 28.13 & 27.79 & 27.40 & 26.58 \\
snsim      & 26.48 & 28.34 & 28.17 & 27.85 & 27.46 & 26.63 \\
\hline
\multicolumn{7}{c}{{\bf median seeing [arcsec]}} \\
\hline
SMTN-002 / snsim  &  0.92 &  0.87 &  0.83 &  0.80 &  0.78 &  0.76 \\
\hline
\multicolumn{7}{c}{{\bf Dark sky [AB mag / arcsec$^2$]}}   \\
\hline
SMTN-002   & 22.95 & 22.24 & 21.20 & 20.47 & 19.60 & 18.63 \\ %% line 1 of Table 2
snsim      & 22.98 & 22.23 & 21.19 & 20.46 & 19.60 & 18.61 \\
\hline
\multicolumn{7}{c}{{\bf NEA [pixel$^2$]}}   \\
\hline
snsim (Moffat, $\beta=7$)     & 58.8  & 52.7  & 48.0  & 44.7  & 42.6  & 40.5  \\
\hline
\multicolumn{7}{c}{{\bf Limiting mag ($5 \sigma$), 30-s visit}}   \\
\hline
SMTN-002                    &  23.60     &  24.83     &  24.38     &   23.92    &  23.35     &  22.44  \\
snsim (Moffat, $\beta=7$)   &  23.61     &  24.83     &  24.35     &   23.88    &  23.30     &  22.43  \\
\hline
\end{tabular}
\end{center}
\end{table}


\begin{figure}[t]
\begin{center}
\subfigure[LSE-40]{\includegraphics[width=0.45\linewidth]{lse_40_passbands.pdf}}
\subfigure[SMTN-002]{\includegraphics[width=0.45\linewidth]{smtn002_passbands.pdf}}
\caption{Instrument passbands}
\end{center}
\end{figure}


\begin{figure}[t]
\begin{center}
\includegraphics[width=\linewidth]{lsst_model_summary.pdf}
\caption{Zero-points, median seeing, dark sky mags and limiting mags}
\label{fig:lsst_model_summary}
\end{center}
\end{figure}


\section{Additional figures}

\begin{figure}[t]
  \begin{center}
    \includegraphics[width=\linewidth]{metric_DD_744.pdf}
    \caption{}
  \end{center}
\end{figure}


\begin{figure}[t]
  \begin{center}
    \includegraphics[width=\linewidth]{metric_DD_1427.pdf}
    \caption{}
  \end{center}
\end{figure}

\begin{figure}[t]
  \begin{center}
    \includegraphics[width=\linewidth]{metric_DD_2412.pdf}
    \caption{}
  \end{center}
\end{figure}

\begin{figure}[t]
  \begin{center}
    \includegraphics[width=\linewidth]{metric_DD_2786.pdf}
    \caption{}
  \end{center}
\end{figure}



\begin{figure*}[t]
  \begin{center}
    \includegraphics[width=\linewidth]{metric_WFD_660.pdf}
    \caption{}
  \end{center}
\end{figure*}

\begin{figure*}[t]
  \begin{center}
    \includegraphics[width=\linewidth]{metric_WFD_1088.pdf}
    \caption{}
  \end{center}
\end{figure*}




\end{document}
% ======================================================================
% 








% There are a number of useful \LaTeX\xspace commands predefined in
% \code{macros.tex}.  Notice that the section labels are prefixed with
% \code{sec:} to allow the use of the \verb=\secref= command to
% reference a section (\ie, \secref{intro}).  Figures can be referenced
% with the \verb=\figref= command, which assumes that the figure label
% is prefixed with \code{fig:}.  In \figref{example} we show an example
% figure.  You'll notice that the actual figure file is found in the
% \code{figures} directory.  However, because we have specified this
% directory in our \verb=\graphicspath= we do not need to explicitly
% specify the path to the image.

% The \code{macros.tex} package also contains some conventional
% scientific units like \angstrom, \GeV, \Msun, etc. and some editorial
% tools for highlighting \FIXME{issues}, \CHECK{text to be checked},
% \COMMENT{comments}, and \NEW{new additions}.
