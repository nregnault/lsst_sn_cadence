% 
% ======================================================================
\RequirePackage{docswitch}
% \flag is set by the user, through the makefile:
%    make note
%    make apj
% etc.
\setjournal{\flag}

\documentclass[\docopts]{\docclass}

% You could also define the document class directly
%\documentclass[]{emulateapj}

% Custom commands from LSST DESC, see texmf/styles/lsstdesc_macros.sty
\usepackage{lsstdesc_macros}

\usepackage{graphicx}
\graphicspath{{./}{./figures/}}
\bibliographystyle{apj}
\usepackage{subfigure}

\usepackage{draftwatermark}
\SetWatermarkScale{1}
\SetWatermarkLightness{0.90}
\usepackage{multirow}
% Add your own macros here:



\newcommand{\fia}{fielda}
\newcommand{\fib}{fieldb}
\newcommand{\fic}{fieldc}
\newcommand{\fiap}{fielda'}
\newcommand{\fibp}{fieldb'}
\newcommand{\ficp}{fieldc'}

% 
% ======================================================================

\begin{document}

\title{ On the cadence of the LSST SN survey(s) }

\maketitlepre

\begin{abstract}

  % We discuss key design elements that must be taken into account to
  % maximize the impact of the LSST deep and wide SN surveys. We show
  % that they can be compressed into simple requirements on the quality
  % of the SN light curves.
  
  We describe a simple metric that allows one to evaluate the quality of
  the SN~Ia light curves that will be delivered by the LSST Wide and
  DDF surveys.  Evaluating this metric does not require to generate
  and fit SN lightcurves.  It can therefore be used to evaluate
  efficiently years of survey operations. It simple enough to be
  included in the survey operation scheduler.
  
  Using this approach, we evaluate a subset of the \code{OpSim}
  \code{Minion\_1016} cadence.  For the DDF fields, we find that the
  cadence and depth simulated in \code{Minion\_1016} do not allow to
  build a redshift limited survey up to $z \sim 1$. For the main
  survey, the depth of the 30-s visit satisfies the requirements,
  expect in $z$.  We are currently evaluating the rolling cadence that
  was suggested for the wide fields.

  The results presented in this work depend heavily on our knowledge
  of (1) the throughput of the LSST telescope and camera system and
  (2) the median observing conditions at Cerro Pachon (sky brightness
  and seeing).  We compare the two documented LSST instrument models
  (\code{LSE-40} and \code{SMTN-002}) with \code{Minion\_1016}.  The
  5-$\sigma$ depth obtained with \code{Minion\_1016} is about 0.8-mag
  lower than what can be inferred using the \code{LSE-40} instrument
  model (and the median seeing and sky brightnes values reported in
  the accompanying paper).  This impacts the effective depth of a LSST
  SN survey, and increases significantly the cost of the LSST SN survey.
  
  The results presented in this note will be updated as new
  \code{OpSim} realizations become available.
\end{abstract}

% Keywords are ignored in the LSST DESC Note style:
\dockeys{latex: templates, papers: awesome}

\maketitlepost

% ----------------------------------------------------------------------
% 
\newpage
\section{Introduction}
\label{sec:intro}


In this note, we examine the cadence envisioned for the 10 years of
LSST operations.  In \S\ref{sec:design_notes} we discuss the key
requirements that drive the design of a SN survey.  We then (\S
\ref{sec:instrument_model_summary}) discuss the models that we used to
assess the flux uncertainties affecting the LSST flux measurements.
In light of the expected 5-$\sigma$ depth of the LSST visits we
evaluate (\S\ref{sec:effective_depth}) the depth of the wide and DDF
SN surveys, and we derive a nominal cadence and depth that would allow
us to build complete samples up to a redshift of 0.4 (wide) and 0.8
(deep). In \S\ref{sec:metric} present a metric that permits to
evaluate efficiently a survey cadence (e.g. from a \code{OpSim} run)
without having to rely on extensive simulations.  In the next three
sections (\S\ref{sec:ddf_cadence}, \S\ref{sec:wide_cadence},
\S\ref{sec:rolling_cadence}), we apply this metric to the \code{OpSim}
DDF, wide and rolling-wide cadences respectiveley. We conclude in
\S\ref{sec:conclusions}.



% ----------------------------------------------------------------------

\section{Design Elements}
\label{sec:design_notes}

With an adapted rolling cadence, LSST has the capability to discover
and follow-up  $10^4$ to $10^5$ SNe~Ia in the redshift range $0.1 < z <
1$ (Wide and DDF surveys combined) and to build a redshift-limited
sample up to $z \sim 1$. The distant part of this Hubble diagram will
be competitive with the DESI constraints. At low redshifts ($z < 0.5$)
LSST will have essentially no competitor.

% Today's SN~Ia Hubble diagrams are systematics dominated.  When
% designing a future high-statistics SN survey, provisions and plans
% must be made to push down as much as possible the level of the (known)
% sources of systematics. Photometric calibration is today the dominant
% contribution to the systematic error budget.  There is ongoing work in
% the Project and in DESC to lower its contribution by a factor $\sim 5$
% w.r.t.  today's standards.

The cosmological impact of a SN survey essentially depends on its
redshift lever-arm, which itself depends almost exclusively on the
redshift limit, $z_{lim}$, beyond which one starts to loose events
because of poor sampling.
%One important systematics that can be eliminated entirely at the
%survey design stage is the Malmquist bias.  It impacts the faint end
%of the wide and DDF surveys ($z \sim 0.4$ and $z \sim 0.9$ resp.)
%where one has to model the fraction of events lost.  
One can of course model the fraction of events lost, however, this
procedure yields uncertainties, that are quite intricated with (1) the
control of the demographic evolution and the intrinsic properties of
SNe~Ia (2) the standardization procedures (see impact on $\beta$).  As
a consequence, the supernovae beyond $z_{lim}$ are of limited
usefulness.  $z_{lim}$ is therefore an essential characteristic of the
survey design. Once we have chosen $z_{lim}$, we need to build the cadence 
that delivers a complete sample up to that redshift limit. 



%% \paragraph{Redshift-limited sample} 

Note that this ``redshift-limit''
is not a detectability limit. It is the redshift value beyond which
we start losing a fraction of the events, because their photometric
follow-up is not good enough to (1) measure a distance and (2) perform
a photometric identification.

A redshift limited sample is a sample such that every SN~Ia that occurs:
(1) in a well defined observer-frame time interval (that corresponds
  to a $\sim 180$ day search season) $\mathrm{[MJD_{start}; MJD_{end}]}$;
(2) in the redshift range $z < z_{lim}$;
(3) in a region of the SN~Ia parameter space that is large enough to
  encompass a potential evolution with redshift of the SN~Ia
  demographic properties;
has a follow-up which is ``good enough'' to (1) identify it
photometrically as a SN~Ia and (2) derive a standardized distance from
its lightcurve.

\begin{figure}[t]
\begin{center}
\includegraphics[width=0.75\linewidth]{sn_parameter_space.pdf}
\caption{JLA supernovae the $(X_1,Color)$ parameter space -- (blue:
  nearby, green: SDSS, orange: SNLS).  }
\label{fig:jla_X1_C}
\end{center}
\end{figure}

To define the parameter space region of interest, we propose to
proceed as follows: up to a good approximation, SNe~Ia form a
2-dimensional family, that may be indexed for example with color and
lightcurve-width (for example, the SALT2-color and the
SALT2-$X_1$-parameter).  On figure \ref{fig:jla_X1_C}, we show the
distribution of the JLA supernovae, in the $(X_1,C)$ parameter
space. We note sizeable differences between the nearby and distant
SNe, in particular in $X_1$.  However, we see that the full JLA sample
is comprised in the region $X_1 = [-3,3], Color= = [-0.3, 0.3]$. 

The core of the distribution is in fact encircled in a smaller region
(red ellipse). Two extreme events have been marked on the same figure:
the red dot corresponds to a faint SN ($X_1=-2, C=0.2$), while the
blue dot correspond to a bright ($X_1=2, C=-0.2$) event. To be
complete up to $z_{lim}$, the survey has to deliver a cadence that
allows to derive a luminosity distance and a photometric
identification for an event of such $X_1$ and Color, at a redshift
$z_{lim}$. 

The much larger LSST samples will contain events that are outside that
region of interest.  However, we can infer from JLA that the core of
the distribution of SN~Ia is well contained our ellipse -- at least at
all redshifts below 1.

In what follows, we will sometimes refer to our ``standard faint SN''
($X_1=-2, C=0.2$) our standard bright SN ($X_1=2, C=-0.2$) and of our,
the average SN ($X_1=0, C=0$). These are the three fiducial events
that need to be studied in detail to assess the cadence of a the
survey.


\paragraph{Lightcurve quality requirements} Light curve quality
requirements have been discussed a few years ago, for the LSST-Euclid
paper \citep{2014A&A...572A..80A}. We summarize them below:

\begin{itemize}
\item the follow-up of each supernova must be good enough in the
  observer-frame bands that correspond to the $B$- and $V$-restframe
  spectrum ($3800 \angstrom < \lambda < 7000 \angstrom$).  At
  high-redshift, in particular, one should avoid relying on the $UV$
  restframe region to derive a distance, given the high intrinsic
  dispersion of SN~Ia at those wavelengths.
  
\item we require the light curve shape to be well sampled in the
  (restframe) phase interval $[-10;+30]$ days, with at least four
  visits before peak (each of those visits in any of the eligible
  band), and ten visits after peak.  To obtain this in the lower
  redshift region of the Hubble diagram, one requires an
  observer-frame cadence of 4 days.  In the upper redshift region (DDF
  fields), this requirement may be slightly relaxed. However, since we
  are going to rely exclusively on photometric identification, it is
  essential to secure a tight sampling of the SN color evolution at
  all redshift.
  
\item we require that the photon noise contribution to the distance
  measurement is subdominant w.r.t. the intrinsic dispersion of the
  SNe (after standardization).  There are several ways to quantify
  this.  With today's standardization techniques, the SN standardized
  distance modulus is:
  \begin{equation}
    \mu = m^\star_B + \alpha X_1 - \beta C - \cal{M}
  \end{equation}
  where $m^\star_B$ is the peak brightness in restframe $B$, $X_1$
  characterize the lightcurve width, and $C$ is an estimate of the
  restframe color $B-V$. $\alpha$, $\beta$ and $\cal{M}$ are global
  parameters, fit along with the cosmology. If the light curve is
  correctly sampled (see point above), the propagation of the
  measurement uncertainties affecting $m^\star_B$, $X_1$ and $C$ is
  dominated by the contribution of $\sigma_C$. 
  
  The dominant contribution is carried by the color (since
  $\beta \sim 3$). This means that requiring $\sigma C < 0.04$ ensures
  that $\sigma \mu < 0.1$, below the intrinsic dipersion in the Hubble
  diagram, after standardization.

\item each SN must have good quality measurements in at least three
  bands. We need two bands, covering the restframe $B$ and $V$ region,
  to constrain the restframe color of the SN. We need to provision an
  additional band (redder than restframe $V$), to enable next
  generation standardization techniques, that will likely rely on two
  restframe colors.
\end{itemize}


\paragraph{Conclusion} As a conclusion, evaluating the cadence boils
down to evaluating the follow-up of the faintest SN~Ia ($(X_1=-2,
C=0.2$) around $z = z_{lim}$. Our goal is to make sure that the light
curves of this SN of such events pass these requirements, throughout
the whole search season.

The requirements above fall into two categories: 
\begin{itemize}
\item we have  cadence requirements, which are easy  to evaluate (just
  count the number of visits before  and after peak in each band).  We
  require  a  restframe  cadence  of  3  days  or  better.   Following
  \cite{2014A&A...572A..80A}, we require an  observer frame cadence of
  4 days.

\item we also have depth requirements.  To quantify this, we can use
  the SNR on the SALT2 color, evaluated from the fit of the light
  curve of the faintest SN~Ia at different redshift and survey epochs:
  we require $\sigma_C < 0.04$ for all such SNe.  An equivalent
  approach consists in placing requirements on the uncertainty of the
  amplitude of the light curve shape.  
\end{itemize}

% With three bands, we reach $\sigma_c < 0.04$ by requiring a SNR of 20
% or better on the light curve amplitude, in each band separately.  It
% is somewhat simpler to evaluate, as it does not require to perform a
% LC fit (or fisher matrix estimate) with a SN LC model.

Before we evaluate the OpSim cadence, we need to assess on some kind
of simple, optimal, cadence what the LSST can deliver in terms of
limiting redshift. To do that, we first need to have a look in the
instrument and observing condition models that enter into the SNR and
5-$\sigma$ depth computations. This is the subject of the next
section.



% ----------------------------------------------------------------------
\section{Instruments models \& Observing conditions}
\label{sec:instrument_model_summary}

Several models of the LSST throughput have been pusblished.  The
forecasts published in \cite{2014A&A...572A..80A} used the model
described in \cite[][hereafter LSE-40]{LSE-40}. 

\code{LSE-40} has been revised recently. The most recent model (as far
as we know) is presented in \cite[][hereafter SMTN-002)]{SMTN-002}.
The details of \code{LSE-40} and \code{SMTN-002} are discussed in
appendix \ref{sec:lsst_instrument_models}.  In particular, the model
characteristics are listed in tables \ref{tab:lse40} and
\ref{tab:smtn002}, the model throughputs are compared on figure
\ref{fig:throughput_comparison} and the throughput ingredients (mirror
reflectivities, filters ...) on figure \ref{fig:passband_constituents}.

We note that the throughput of \code{SMTN-002} is almost 40\% lower
than that of \code{LSE-40}.  While the latter model was probably a
little optimistic, \code{SMTN-002} seems a bit pessimistic for an
imager.  We have not been able to find, in the LSST literature, a
discussion on the differences between \code{SMTN-002} and
\code{LSE-40}.  Since the instrument throughput is an essential
ingredient of the survey depth, it is important to understand whether
\code{SMTN-002} was built as a realistic model, or as some kind of a
worst case assessment (as of today, we do not know).

On tables \ref{tab:lse40} and \ref{tab:smtn002} of appendix
\ref{sec:lsst_instrument_models}, we have reported the observing
conditions (median seeing and dark sky background) expected by in the
\code{LSE-40} and \code{SMTN-002} references.  They differ
significantly, \code{SMTN-002} being again more pessimistic than
\code{LSE-40}.

\code{Minion\_1016} uses \code{SMTN-002} (or a very similar instrument
model), but predicts larger median seeing values.  The median sky
obtained from the \code{Minion\_1016} logs is (expectedly) brigher
than the dark sky levels of \code{LSE-40} and \code{SMTN-002}. In the
$z$ band, the Minion sky levels are high, compared to what is measured
by DES.  In the $y$-band, the Minion prediction is strictly equal to
17.3 and shows no variation with the moon phase.  We believe that the
\code{Minion\_1016} sky level is not valid in the $y$-band. As we will
see in the following of this note, this affects our ability to predict
the effective depth of the DDF survey. 

\begin{figure}[t]
\begin{center}
\includegraphics[width=\linewidth]{lsst_model_summary.pdf}
\caption{Zero-points, median seeing, dark sky mags and limiting mags}
\label{fig:lsst_model_summary}
\end{center}
\end{figure}


On figure \ref{fig:lsst_model_summary}, we compare the throughtput
(ZP), median seeing and sky of LSE-40, SMTN-002 and
\code{Minion\_1016}.  On the lower left panel, we estimate the typical
5-$\sigma$ depth of a 30-second LSST visit, from (1) the \code{LSE-40}
and \code{SMTN-002} instrument model and observing conditions, and (2)
the \code{SMTN-002} instrument model and Minion median observing
conditions. We note that the median depth predicted by \code{Minion}
is 0.8 to 1 mag lower that what was inferred from LSE-40 in the
\cite{2014A&A...572A..80A} study. As we will see below, this has a
severe impact on the cost of the LSST SN survey -- of the DDF survey
in particular.

As a conclusion, the results presented below will need to be updated
as our understanding of the instrument model and observing conditions
at Cerro Pachon improves. 




% ----------------------------------------------------------------------
\section{The effective depth of the DDF and Wide surveys}
\label{sec:effective_depth}

\subsection{DDF fields}

Before we evaluate the \code{Minion\_1016} cadence, let's assess the
quality of the SN follow-up if the survey delivers a nominal 4-day
cadence. We adopt the sky brightness values and median seeing values
that are reported in \cite{LSE-40} and \cite{SMTN-002} respectively.
The computations are performed with a SN simulator derived from the
one that was used to carry out the forecasts for a combined
LSST-Euclid survey \cite{2014A&A...572A..80A}.


\begin{figure*}
\begin{center}
\subfigure[\code{SMTN-002} -- 600-s visits]{\includegraphics[width=0.49\linewidth]{sigc_lsstpg_ddf_600.pdf}}
\subfigure[\code{SMTN-002} -- 1800-s DDF visits]{\includegraphics[width=0.49\linewidth]{sigc_lsstpg_ddf_1800_cad3.pdf}}
\subfigure[\code{SMTN-002} -- 600-s visits]{\includegraphics[width=0.49\linewidth]{snr_lsstpg_ddf_600.pdf}}
\subfigure[\code{SMTN-002} -- 1800-s DDF visits]{\includegraphics[width=0.49\linewidth]{snr_lsstpg_ddf_1800_cad3.pdf}}
\end{center}
\caption{Upper panels: $\sigma_C$ (shot noise only) obtained on the
  DDF fields as a function of redshift.  On the left, for the standard
  (r: 600-s, i: 600-s, z: 780-s, y: 600-s) visits and a 4 day cadence,
  on the right for 1200-s visits (same cadence). Lower panels: SNR
  obtained on the light curve amplitude, as a function of
  redshift. Left and right: same observing conditions.}
\label{fig:sigc_vs_z_smtn002}
\end{figure*}

The quality of the distances is a function of the resolution we get on
the SN color.  On figures \ref{fig:sigc_vs_z_smtn002} and
\ref{fig:sigc_vs_z_lse40} we display how $\sigma_C$ varies with
redshift, for the average SN (black curve), and our faint and bright
fiducial SNe (red and blue curves respectively).  


The left panel of figure \ref{fig:sigc_vs_z_smtn002} shows what we
obtain with a 4-day cadence and the SMTN-002 instrument model.  The
sharp increase of $\sigma_C$ around $z \sim 0.7$ corresponds to when
the $r$-band starts sampling the $\lambda < 3600\AA$ region. As
discussed above, SNe are much more variable in the UV, hence we cannot
rely on this spectral region to derive distances (until somebody
figures out how to standardize SNe in the UV).

We see that once the $r$-band has been dropped, we do not have enough
photostatistics from the $i,z,$ and $y$-band combined, to constrain
$C$ below our target $\sigma_C < 0.04$.  Hence, our redshift limit is
of about 0.7 -- only very slightly deeper than SNLS (0.65). As a
consequence, the depth of the DDF rolling survey needs to be increased
in the redder bands. There are two ways to do that: either we increase
the cadence, or we increase the depth (or both).

On the right panel of figure \ref{fig:sigc_vs_z_smtn002} we show what
can be obtained with 1800-s visits (3 times the standard DDF visit)
and a cadence of 3 days. At this cost, we can build a sample complete
up to $z \sim 0.8$. 

On the upper panel of figure \ref{fig:sigc_vs_z_lse40}, we show the
same scenarios, with the LSE-40 model.  This is essentially to
illustrate that it is really important to understand which of
LSE-40 or SMTN-002 is the realistic assessment of the observing
conditions.

The bottom panel shows the SNR on the amplitude of the lightcurves for
our faint fiducial SN. We see that we can reach our $\sigma_C \sim
0.04$ target, as long as we can secure a integrate SNR of 60 in $i$
and $35$ in $z$.  In $y$, obtaining these levels of signal to noise
seems to be very hard. Obtaining $SNR=10$ up to $z\sim 0.8$ seems to
be enough, as long as the two other bands reach the targets listed
above.


% We present 4 different scenarios:
% \begin{description}
% \item[Figure \ref{fig:sigc_vs_z_4_day_cadence}, left panels] standard
%   DDF visits (r: 600-s, i: 600-s, z: 780-s, y: 600-s); 4-day cadence;
%   instrument model and observing conditions taken from
%   \code{SMTN-002}.
% \item[Figure \ref{fig:sigc_vs_z_4_day_cadence}, right panels]
%   increasing the visits to 1200-s ($rizy$); 4-day cadence; instrument
%   model and observing conditions taken from \code{SMTN-002}.]
% \item[Figure \ref{fig:sigc_vs_z_2_day_cadence}, left panels] 1200-s in
%   ($rizy$), but with a 2-day cadence; instrument model and observing
%   conditions taken from \code{SMTN-002}.]
% \item[Figure \ref{fig:sigc_vs_z_2_day_cadence}, left panels] 1200-s in
%   ($rizy$); 2-day cadence; instrument model and observing conditions
%   taken from \code{LSE-40}.]
% \end{description}

\begin{figure*}
\begin{center}
\subfigure[\code{LSE-40} -- 600-s -- 4 day cadence]{\includegraphics[width=0.49\linewidth]{sigc_lsst_ddf_600.pdf}}
\subfigure[\code{LSE-40} -- 1800-s -- 3 day cadence]{\includegraphics[width=0.49\linewidth]{sigc_lsst_ddf_1800_cad3.pdf}}
\subfigure[\code{LSE-40} -- 600-s -- 4 day cadence]{\includegraphics[width=0.49\linewidth]{snr_lsst_ddf_600.pdf}}
\subfigure[\code{LSE-40} -- 1800-s -- 3 day cadence]{\includegraphics[width=0.49\linewidth]{snr_lsst_ddf_1800_cad3.pdf}}
\end{center}
\caption{Same as figure \ref{fig:sigc_vs_z_smtn002} with the LSE-40
  model.  }
\label{fig:sigc_vs_z_lse40}
\end{figure*}

These figures are a representative subset of what has been explored so far. 

We see that pushing the completeness limit to $s \sim 0.9$ seems to be
essentially inaccessible.  Obtaining a survey complete up to $z \sim
0.8$ or more requires at least tripling the exposure time and securing
a high cadence of about 3 days.

Table \ref{tab:nominal_scenario_DDF} summarizes the cadence, exsposure
times and the average $5-\sigma$ depth per visit. We propose this a a
``nominal'' cadence and depth for the DDF. We note that implementing
it requires a partial recadencing similar to what has been proposed
for the wide component of the survey.

\begin{table*}[t]
\begin{center}
\caption{A nominal scenario for the DDF that allows to build a SN
  sample complete up to $z \sim 0.75$.}
\label{tab:nominal_scenario_DDF}
\begin{tabular}{l|cccc}
\hline
\hline
              & $r$ & $i$ & $z$ & $y$ \\
\hline 
$T_{exp}$      & 1200 & 1800 & 1800 & 1800 \\
$m_{5\sigma}$  & 26.43    & 26.16    &  25.56    &  24.68   \\
cadence       &  \multicolumn{4}{c}{3 days} \\
Target amplitude SNR & $>25$ & $>60$ & $>35$ & $>20$ \\
\hline
\end{tabular}
\end{center}
\end{table*}

Finally, we note the impact of changing the instrument model on the
effective depth of the survey: with the same time budget, we increase
very sizeably the depth of the survey (from $z \sim 0.75$ to $z\sim
0.9$), with a significant increase of the quality of the average
lightcurve.  We think it is essential for the SNWG members to
understand the origin of the differences between \cite{LSE-40} and
\cite{SMTN-002}.


\subsection{Wide fields}

We have performed a similar study for the wide fields. We show on
figure \ref{fig:sigc_vs_z_wide} the evolution of
$\sigma_C$ for the standard visits (30-s in $griz$) and a cadence of 4
days. On the right panels, we show the same, figures, with a cadence
slightly improved (3 days).

\begin{figure*}[t]
\begin{center}
\subfigure[\code{SMTN-002} -- standard visits]{\includegraphics[width=0.49\linewidth]{sigc_lsstpg_wide_30.pdf}}
\subfigure[\code{SMTN-002} -- deeper]{\includegraphics[width=0.49\linewidth]{sigc_lsstpg_wide_30_cad3.pdf}}
\subfigure[\code{SMTN-002} -- standard visits]{\includegraphics[width=0.49\linewidth]{snr_lsstpg_wide_30.pdf}}
\subfigure[\code{SMTN-002} -- deeper]{\includegraphics[width=0.49\linewidth]{snr_lsstpg_wide_30_cad3.pdf}}
\end{center}
\caption{Upper panels: $\sigma_C$ (shot noise only) obtained on the
  standard fields as a function of redshift.  On the left, for the
  standard 30-s visits and a 4 day cadence. On the right, we double
  the exposure times in $r$ and $i$.  (same cadence).  Lower panels:
  SNR obtained on the light curve amplitude, as a function of
  redshift.}
\label{fig:sigc_vs_z_wide}
\end{figure*}

We see that is is relatively easy to reach a completeness limit of $z
\sim 0.4$, with the standard visits and a cadence of 3 to 4 days
(observer frame).  We summarize our nominal target cadence in table
\ref{tab:nominal_scenario_wide}. Whether the survey can deliver such a
cadence for the wide will be discussed in sections
\ref{sec:wide_cadence} and \ref{sec:rolling_cadence}, that will be
based on realistic \code{OpSim} simulations.


\begin{table*}[t]
\begin{center}
\caption{A nominal scenario for the wide that allows to build a SN
  sample complete up to $z \sim 0.4$.}
\label{tab:nominal_scenario_wide}
\begin{tabular}{l|cccc}
\hline
\hline
              & $g$ & $r$ & $i$ & $z$ \\
\hline 
$T_{exp}$      & 30       &   30    &  30        & 30       \\
$m_{5\sigma}$  &  24.83   &  24.35   &  23.88    &  23.30   \\
cadence       &  \multicolumn{4}{c}{3 days} \\
Target amplitude SNR & $>30$ & $>40$ & $>30$ & $>20$ \\
\hline
\end{tabular}
\end{center}
\end{table*}


Now, we will present simple metrics, that can be turned into powerful
tools to evaluate \code{OpSim} cadences without having to explicitely
simulate and fit supernova lightcurves.  They all are based on the
Fisher matrix of the (monoband) light curve fit to evaluate the total
SNR we obtain on the amplitude of the SN lightcurve.



% ----------------------------------------------------------------------
\section{Two useful metrics to evaluate a cadence}
\label{sec:metric}


\subsection{The signal-to-noise on the light-curve amplitude}

If we fit a model $A \times \ell(t)$ on a lightcurve $(t_i, y_i)$, the
least square estimate of the amplitude is:
\begin{equation}
  \hat{A} = \frac{\sum_i w_i \ell_i y_i}{\sum_i w_i \ell^2_i}
\end{equation}
where we note $\ell(t_i) = \ell_i$, and $w_i = 1 / \sigma_i^2$ the
measurement weights. The variance of $\hat{A}$ is:
\begin{equation}
  \mathrm{Var}(\hat{A}) = \left(\sum_i w_i \ell^2_i\right)^{-1}
\end{equation}
and signal to noise we get on the amplitude $\hat{A} /
\sigma_{\hat{A}}$ is:
\begin{equation}
  SNR = \left(\sum_i w_i L^2_i\right)^{1/2}
\end{equation}
where $L_i = A \times \ell_i$.  The weights can be expressed as a
function of the 5-$\sigma$ limiting flux of each visit $i$, $f_{i|5}$:
\begin{equation}
  SNR = 5 \times \left(\sum_i L^2_i f^{-2}_{i|5}\right)^{1/2}
  \label{eqn:snr}
\end{equation}
This metrics is easy to compute, all it takes is a tabulated model of
the light curve shape, and a cadence file, containing, for each visit,
an assessment of the $5\sigma$-depth reached during that visit. 

In tables \ref{tab:nominal_scenario_DDF} and
\ref{tab:nominal_scenario_wide}, we have specified target SNR for the
light curve amplitude (in each band).  It is then relatively easy to
implement formula \ref{eqn:snr} in the survey scheduler, to evaluate
whether a light curve at some phase (say +30 restframe days) has
integrated enough photostatistics to reach the SNR target in a given
band -- and alter the observation priority accordingly.


\subsection{A global metric to evaluate the quality of a search \& follow-up season}

The metric above is ``local'', in the sense that it allows to evaluate
the cadence and depth delivered by the survey at a given time. It is
extremely useful to check the regularity of the observations and react
in real time. 

Now, we would like to present another metric that could allow to
evaluate the quality of the cadence and depth delivered by the survey
on an entire season. 

Let's define $\delta(t_i) = \delta_i$, which is equal to the number of
observations in an interval $\Delta t$ around $t_i$ (we express
$\delta_i$ in day$^{-1}$). The metric above can be rewritten:
\begin{equation}
  SNR = 5 \times \left(\sum_i \delta_i L^2_i f^{-2}_{i|5} \Delta t\right)^{1/2}
\end{equation}
where the sum runs on all the $\Delta t$ bins, in a observer frame
time interval covering the supernova lightcurve evolution. 

Let's define $f_{|5}$, the average 5-$\sigma$ limiting flux per visit,
in the band under consideration. The requirement above can be
rewritten:
\begin{equation}
  f_{|5} \left<\delta_i\right>^{-1/2} \leq \frac{5 \sqrt{\Delta t} \sqrt{\sum_i L_i^2}}{SNR}
  \label{eqn:global_metric}
\end{equation}
where $\left<\delta_i\right>$ is the cadence, weighted by the light
curve shape squared: $\left<\delta_i\right> = \sum_i \delta_i
L_i^2/\sum_i L_i^2$.  This sets a limit (in $\frac{e^-}{s}
\sqrt{day}$) on the product of the limiting flux times inverse square
root of the cadence.


\begin{table}[t]
\begin{center}
\caption{cadence-depth limit for a $[X_1=-2, C=0.2]$ SN~Ia}
\label{tab:cadence_depth_limit}
\begin{tabular}{l|c|rrrrr}
\hline
\hline
 &   redshift   &      $g$         &       $r$         &     $i$           &      $z$        &      $y$           \\
 &              &      \multicolumn{5}{c}{$[\mathrm{e^-/s \times \sqrt{day}}]$} \\
\hline
  \multicolumn{2}{l}{Target SNR}  &     30  &      40 &      30 &      20 & \\
\hline
 \multirow{4}{*}{wide}  &     0.2  &     291 &     264 &     335 &     299 & \\
                        &     0.3  &     117 &     160 &     139 &      97 & \\
                        &     0.4  &      63 &      90 &      80 &      32 & \\
                        &     0.5  &      37 &      55 &      44 &      12 & \\
\hline
  \multicolumn{2}{l}{Target SNR} &  & 25 &     60  &  35     &      20   \\
\hline
  \multirow{7}{*}{DDF}                       &     0.5  &    & 59 &      32 &      22 &      29   \\
                       &     0.6  &    & 33 &      23 &      14 &      18   \\
                       &     0.7  &    & 18 &      14 &      11 &      14   \\
                       &     0.8  &    &    &      10 &       8 &      11   \\
                       &     0.9  &    &    &       8 &       5 &       7   \\
                       &     1.0  &    &    &       7 &       3 &       5   \\
                       &     1.1  &    &    &       6 &       2 &       5   \\


     % 0.2  &     386 &     526 &     470 &     251 &     103   \\
     % 0.3  &     123 &     207 &     193 &     150 &      52   \\
     % 0.4  &      39 &     110 &     109 &      84 &      33   \\
     % 0.5  &      15 &      64 &      59 &      51 &      27   \\
     % 0.6  &         &      35 &      38 &      36 &      17   \\
     % 0.7  &         &      18 &      28 &      23 &      13   \\
     % 0.8  &         &         &      21 &      16 &      10   \\
     % 0.9  &         &         &      14 &      13 &       7   \\
     % 1.0  &         &         &       9 &      10 &       5   \\
     % 1.1  &         &         &       5 &       8 &       4   \\
 
     % 0.2  &     916 &    1330 &    1262 &     704 &     289   \\
     % 0.3  &     284 &     513 &     495 &     421 &     149   \\
     % 0.4  &      86 &     267 &     279 &     224 &      91   \\
     % 0.5  &      31 &     153 &     148 &     131 &      76   \\
     % 0.6  &      12 &      81 &      94 &      92 &      43   \\
     % 0.7  &         &      42 &      69 &      57 &      34   \\
     % 0.8  &         &      17 &      52 &      39 &      25   \\
     % 0.9  &         &      11 &      32 &      31 &      17   \\
     % 1.0  &         &         &      20 &      26 &      12   \\
     % 1.1  &         &         &      11 &      20 &      11   \\
\hline
\end{tabular}
\end{center}
\end{table}

Again, this simple metric allows one to estimate the (global) power of
the cadence and depth delivered over one season for one given field,
without having to generate and fit supernova light curves.  Of course,
a supernova scientist is still needed, to compute the numerical values
of the upper limits that appears in equation \ref{eqn:global_metric}
For the records, we report these values in table
\ref{tab:cadence_depth_limit}. They have been computed with
\code{snsim} based on \code{SALT-2.4} and the \code{SMTN-002}
instrument model.

In the next sections, we show how the most recent \code{Minion\_1016}
cadence compares to our cadence-depth requirements, using the metric
above.  



% ----------------------------------------------------------------------
\section{The deep SN survey on the DDF fields}
\label{sec:ddf_cadence}


\begin{figure*}[t]
\begin{center}
\subfigure[$r$]{\includegraphics[width=0.48\linewidth]{m5_cadence_limits_r.pdf}}
\subfigure[$i$]{\includegraphics[width=0.48\linewidth]{m5_cadence_limits_i.pdf}}\\
\subfigure[$z$]{\includegraphics[width=0.48\linewidth]{m5_cadence_limits_z.pdf}}
\subfigure[$y$]{\includegraphics[width=0.48\linewidth]{m5_cadence_limits_y.pdf}}
\caption{The cadence {\em vs.} depth requirements for the DDF LSST SN
  survey in the $r,i,z$ and $y$-bands. The color scale corresponds to
  the metric described in equation \ref{eqn:global_metric}.  The lines are
  the contour-levels computed for the limits indicated on table
  \ref{tab:cadence_depth_limit}. To fulfill the SNR requirements at a
  given redshift, one has to be {\em below} the corresponding
  line. The stars indicate an ambitious -- but attainable -- goal for
  a DDF survey.  The red and blue crosses show what the survey can currently
  deliver according to \code{Minion\_1016}.  Red crosses correspond to
  a the sky model implemented in \code{OpSim v3.3.5} and blue crosses to a
  new sky model that will be implemented in the next versions of \code{OpSim}. }
\label{fig:m5_cadence_limits_ddf}
\end{center}
\end{figure*}

The constrains described in equation \ref{eqn:global_metric} can be
represented graphically in the $m_{5\sigma}$-cadence plane. We do this
on figure \ref{fig:m5_cadence_limits_ddf}, for the bands of interest
for an intermediate and/or deep survey ($rizy$).

This plot allows one to quantify the nominal survey cadence and depth
that are required to operate a survey up to a given redshift limit.
The isocontours correspond to various relevant redshift limits. To
match the requirements needed to reach a given redshift limit, we need
to be {\em below} the corresponding isocontour. As can be seen from
equation \ref{eqn:global_metric}, we can play on the cadence or depth,
to reach a given limit. For example, we may relax the cadence
requirements, at the price of increasing the depth of the survey.

On the same plots, we have represented with red stars the nominal
cadence and depth listed in table \ref{tab:nominal_scenario_DDF}.
Note than this nominal scenario is something that is possible at the
price of tripling the standard DDF exposure times (and achieving an
average inverse cadence of 3 days).



\paragraph{Where we are now, with \code{Minion\_1016}} We have shown,
on the same graph, the median cadence and depth computed for each DDF
field, from the \code{Minion\_1016} sequences. We have 5 DDF, each
spanning 10 search seasons (ten years of operations).  This gives 50
cadence-depth realizations, which we represent with red and blue
crosses correspond to two sky models : the one implemented in
\code{OpSim v3.3.5} \cite{1991PASP..103.1033K} and a new sky model
that will be available for the next \code{OpSim} versions.

Again, to match the cadence-depth requirements at a given redshift, we need
to be {\em below} the corresponding line {\em in the $i$, $z$ and $y$
  bands simultaneously}. We see that the average cadence delivered by
the survey is about a factor 2 below our target. In the $i$ and $z$
bands, the depth is adequate to obtain a survey complete up to $z \sim
0.8$, however, the predicted $y$-band depth is far below what is
needed.


\begin{figure*}[t]
  \begin{center}
    \includegraphics[width=\linewidth]{metric_DD_290.pdf}
    \caption{Colored lines: the SNR of the estimated amplitude of a
      $[X_1=-2, C=0.2]$ SN~Ia at a redshift $z = 1$ as a function of
      peak MJD, computed for the \code{Minion\_1016} cadence and depth
      for field \#290. The SNR target is shown with the dashed red
      line.  Thin gray lines: the cadence delivered in that band,
      averaged in 30-day a sliding window.  The cadence target is
      represented as a dotted line. The plot extends over the 10 years
      of survey operations. The gray area corresponds to the search
      seasons for that field. The light gray area corresponds to the
      margins for which the SN has no points before (resp after)
      peak.}
    \label{fig:snr_metric}
  \end{center}
\end{figure*}


\paragraph{SN specific metric} The global metric above is very
efficient to assess a nominal cadence and depth, and to evaluate a
whole \code{OpSim} run. We also want to check the regularity of the cadence
over a season. This can be done by computing, in each relevant band,
the SNR that can be obtained on the amplitude of the lightcurve
(following equation \ref{eqn:snr}), as a function of the SN peak date (the first metric presented in the previous section).

This is what we show  on figure \ref{fig:snr_metric}, for our fiducial
faint SN at a  redshift $z=1$, for one of the DDF  fields, and for the
whole duration of  the survey operations. As we can  see, we are below
our target SNR of  20 in all the relevant bands. We  also see that the
cadence and depth  is not constant over one single  season. We suggest
that adding  such a  metric in  the scheduler  would allow  to control
finely the depth of the survey  over the whole duration of the season.
The goal would  be to reach our target  SNR shortly\footnote{say, $-15
  \times (1  + z_{lim})$ observer-frame  days} after the  beginning of
the first observations of the season,  and to schedule the cadence and
depth of the visits in such a  way that the SNR stays above the target
SNR in all relevant bands during the season.




\paragraph{~} Our (temporary) conclusions for the DDF fields are that:
\begin{itemize}
\item in all bands, the cadence is about a factor 2 to 3 lower than our target. 
\item the season-to-season dispersion of the cadence is
  significant. It could be reduced if one monitors the recent cadence
  and depth within the scheduler, and tune the cadence and depth
  accordingly.
\item at the cadence delivered by \code{OpSim}, the $i$ and $z$ depth
  is 0.5-mag and 0.25-mag lower than our target (resp). In the
  $y$-band the depth is 1.75-mag below our target, but this is
  probably due to the fact that the Minion sky level is incorrect in that band.
\item the cadence and depth vary quite significantly as a function of
  time, within a season. Again, computing on the fly the amplitude SNR
  for the [$X_1=-2, C=0.2$] SN at the targeted limiting redshift, and
  making sure that it stays above 20 in the $izy$ bands, would allow
  to to stay above $z_{lim}$ for the whole duration of the search
  season.
\end{itemize}



% ----------------------------------------------------------------------
\section{The Wide LSST SN survey}
\label{sec:wide_cadence}

We have produced similar plots for the main (shallow, wide) survey
(before recadencing). Figure \ref{fig:m5_cadence_limits_wide} shows
the cadence {\em vs.}  $m_{5\sigma}$ plane for the wide survey.  The
red stars indicate the nominal scenario outlined in table \ref{tab:nominal_scenario_wide}.

The red and blue crosses indicate what we get from \code{Minion\_1016}
for all  wide fields. Red crosses correspond to the sky model implemented in
\code{OpSim v3.3.5} \cite{1991PASP..103.1033K} and the blue crosses to
a new sky model that will be available for the next \code{OpSim} versions.  If we target a survey complete up to $z_{lim} \sim 0.4$
we find that:
\begin{itemize}
  \item in $r, i, $ and $z$ the cadence is too high ($\sim 10$ w.r.t 4 days observer frame)
  \item in all bands, the cadence is extremely variable from field to
    field (and from one year to another).
  \item in $r$ and $i$, we go deep enough with the standard visits;
  \item in $z$, we are not deep enough. It is likely that most $z$
    band observations are taken with the moon up. For a SN survey targeting 
    $z_{lim} = 0.4$, we need to either increase the exposure time in $z$, and 
    observe also when the moon is down.    
  \item in $g$, the cadence is extremely variable. This is ok for SNe
    close to $z_{lim} \sim 0.4$ as those do not make use of the $g$.  However, 
    this reduces our ability to produce a decent follow-up of the nearby SNe.
\end{itemize}

\begin{figure*}[t]
\begin{center}
\subfigure[$g$]{\includegraphics[width=0.48\linewidth]{m5_cadence_limits_wide_g.png}}
\subfigure[$r$]{\includegraphics[width=0.48\linewidth]{m5_cadence_limits_wide_r.png}}\\
\subfigure[$i$]{\includegraphics[width=0.48\linewidth]{m5_cadence_limits_wide_i.png}}
\subfigure[$z$]{\includegraphics[width=0.48\linewidth]{m5_cadence_limits_wide_z.png}}
\caption{The cadence {\em vs.} depth requirements for the Wide survey,
  in the $g, r, i$ and $z$-bands. The color scale corresponds to the
  metric described in equation \ref{eqn:global_metric}.  The lines are
  the contour-levels computed for the limits indicated on table
  \ref{tab:cadence_depth_limit}. To fulfill the SNR requirements at a
  given redshift, one has to be {\em below} the corresponding
  line. The stars indicate an ambitious -- but attainable -- goal for
  a DDF survey.  The red and blue crosses show what the survey can currently
  deliver according to \code{Minion\_1016}. Red crosses correspond to
  a the sky model implemented in \code{OpSim v3.3.5} and blue crosses to a
  new sky model that will be implemented in the next versions of OpSim.}
\label{fig:m5_cadence_limits_wide}
\end{center}
\end{figure*}

\begin{figure*}[t]
  \begin{center}
    \includegraphics[width=\linewidth]{metric_WFD_309.pdf}
    \caption{Colored lines: the SNR of the estimated amplitude of a
      $[X_1=-2, C=0.2]$ SN~Ia at a redshift $z = 0.4$ as a function of
      peak MJD, computed for the \code{Minion\_1016} cadence and depth
      for field \#309. The SNR target is shown with the dashed red
      line.  Thin gray lines: the cadence delivered in that band,
      averaged in 30-day a sliding window.  The cadence target is
      represented as a dotted line. The plot extends over the 10 years
      of survey operations. The gray area corresponds to the search
      seasons for that field. The light gray area corresponds to the
      margins for which the SN has no points before (resp after)
      peak.}
    \label{fig:snr_metric_wide}
  \end{center}
\end{figure*}

On figure \ref{fig:snr_metric_wide}, we show again the SNR of our
faint SN at $z_{lim} \sim 0.4$ as a function of SN peak MJD in bands
$g, r, i$ and $z$. We see again that the depth in $z$ (and $g$) does
not yet allow to stay above a SNR of 20 over the duration of each
observing season.



% ----------------------------------------------------------------------

\section{Rolling Cadence on the wide survey}
\label{sec:rolling_cadence}

The basic idea of the rolling cadence is to gather the ten-year
observations of a WFD field into a smaller subset of seasons
(typically three). Since it was not possible to simulate such a
strategy with the \code{OpSim} version (v3.3.5) used to produce the \code{Minion\_1016}
file, we have faked a rolling cadence by merging subsets of three WFD
fields in a coherent way. The basic principle is the following. Let us
consider three sets of observations corresponding to WFD fields,
namely \fia, \fib, and \fic, and a mixing rate m
(0$\leq$m$\leq$1).  Observations corresponding to the rolling cadence
will be denoted \fiap, \fibp, and \ficp. The first season is kept
untouched. Observations related to the second season (and also to
seasons 5 and 8) of \fiap~are composed
of those of \fia~plus m fraction of observations of \fib~plus m fraction
of observations of \fic. Observations of \fibp~and \ficp~are comprised of
the remaining observations of \fib~and \fic. The same procedure is
adopted for the following seasons, with the following permutation :
\fia (\fiap) $\rightarrow$ \fib (\fibp) (season 3, 6 and 9) and \fib (\fibp)
$\rightarrow$ \fic (\ficp) (season 4, 7, and 10). A summary of the combinations
is given in  table \ref{tab:rolling_cadence}. We have made slices of
the (ra,dec) area corresponding to the WFD survey to choose the fields
to be merged. Each ra slice of two degrees was splitted in three dec
regions. The three closest fields (one from each part) were considered for the
merging. 

\begin{table}[t]
\begin{center}
\caption{Procedure used to merge WFD fields to fake a rolling
  cadence. Three fields are used here but one could extend the
  procedure up to nine.The mixing rate m was set to 0.8.}
\label{tab:rolling_cadence}
\begin{tabular}{c|l}
\hline
\hline
    season   &      Observations \\
\hline
       & \fiap = \fia \\
    1 & \fibp = \fib \\
       & \fic = \fic \\
\hline
               & \fiap = \fia+m*\fib+m*\fic \\
    2, 5, 8 & \fibp = (1-m)*\fib \\
               & \ficp = (1-m)*\fic \\
\hline
               & \fiap = (1-m)*\fia \\
    3, 6, 9 & \fibp = \fib +m*\fia+m*\fic\\
              & \ficp = (1-m)*\fic \\
\hline
                 & \fiap = (1-m)*\fia\\
    4, 7, 10 & \fibp = (1-m)*\fib \\
                 & \ficp = \fic +m*\fia+m*\fib\\
\hline
\end{tabular}
\end{center}
\end{table}

The resulting observing strategy is given on figure
\ref{fig:rolling_strategy}. We have tried to merge fields in a coherent way. In addition to
elementary ajustments (such as the (ra,dec) values), we have performed
a reshuffling of observations corresponding to \fiap (seasons 2, 5,
and 8),\fibp (seasons 3, 6, and 9), and \ficp (seasons 4, 7, and
10). For each season and each band, we aimed at distributing
observations while keeping good observing conditions : airmass were
requested to lie in the range [1,1.5] (this upper limit was imposed by \code{OpSim}
when producing the \code{Minion\_1016} cadence), the Moon phase lower
than 60\% and the Moon zenith distance higher than 85 degrees. Observations at twilight were not accepted.

\begin{figure*}[t]
  \begin{center}
    \includegraphics[width=\linewidth]{figures/Rolling_fields.pdf}
    \caption{Observing strategy corresponding to the rolling
      cadence. Blue, red and black crosses correspond to fields
      observed during seasons (2,5,8) (fielda), (3,6,9) (fieldb), and
      (4,7,10) (fieldc), respectively.}
    \label{fig:rolling_strategy}
  \end{center}
\end{figure*}


The results of the procedure used to produce a rolling cadence are
given on figures \ref{fig:rolling_airmass} and
\ref{fig:rolling_cadence_filtre}. The frequency of observations is around 2.5 times higher
in the Rolling survey, with mean cadences of 10.1, 7.3, 3.5, 3.5, 4.4,
4.4 days for u, g, r, i, z and y bands,
respectively (see table \ref{tab:cadence_bands}). A comparison of the
median and mean values of the sky brightness (table
\ref{tab:rolling_sky}) shows a compatibility of the results between
the two strategies, except for the z-band. This may be explained by
observing conditions which are less favourable for \code{Minion\_1016}
in this band. 


\begin{figure*}[t]
  \begin{center}
    \includegraphics[width=\linewidth]{figures/Airmass_vs_mjd_511.pdf}
    \caption{Airmass as a function of time for WFD field number
      511. Blue points correspond to the \code{Minion\_1016}
      observations and red stars to a rolling survey. Field number 511
    is of type 'fielda'.}
    \label{fig:rolling_airmass}
  \end{center}
\end{figure*}

\begin{figure*}[t]
\begin{center}
\subfigure[$u$]{\includegraphics[width=0.48\linewidth]{Cadence_Hist_u.png}}
\subfigure[$g$]{\includegraphics[width=0.48\linewidth]{Cadence_Hist_g.png}}
\subfigure[$r$]{\includegraphics[width=0.48\linewidth]{Cadence_Hist_r.png}}
\subfigure[$i$]{\includegraphics[width=0.48\linewidth]{Cadence_Hist_i.png}}
\subfigure[$z$]{\includegraphics[width=0.48\linewidth]{Cadence_Hist_z.png}}
\subfigure[$y$]{\includegraphics[width=0.48\linewidth]{Cadence_Hist_y.png}}

\caption{Mean cadences for WFD fields of \code{Minion\_1016} (black)
  and for the Rolling cadence (red). Only seasons corresponding to
  higher number of observations (i.e seasons (2,5,8), (3,6,9) and
  (4,7,10) for field types a, b, and c, respectively) have been considered.}
\label{fig:rolling_cadence_filtre}
\end{center}
\end{figure*}


\begin{table}[t]
\begin{center}
\caption{Mean cadences (in days)  for the WFD survey and the Rolling
  Cadence. Only seasons corresponding to
  higher number of observations (i.e seasons (2,5,8), (3,6,9) and
  (4,7,10) for field types a, b, and c, respectively) have been considered. }
\label{tab:cadence_bands}
\begin{tabular}{c|c|c}
\hline
\hline
   Band  & WFD   &      Rolling \\
\hline
     u   & 25.2 $\pm$ 5.8 & 10.1 $\pm$ 2.9\\
     g   &  17.1 $\pm$ 4.8& 7.3 $\pm$ 1.9 \\
     r   &8.4 $\pm$ 2.0 & 3.5 $\pm$ 1.0 \\
     i   &8.4 $\pm$ 2.0  & 3.5 $\pm$ 1.0\\
     z  & 11.5 $\pm$ 2.3 & 4.4 $\pm$ 0.9\\
    y  & 11.8 $\pm$ 2.4& 4.4 $\pm$ 1.0\\
\hline
\end{tabular}
\end{center}
\end{table}

\begin{table}[t]
\begin{center}
\caption{Median and mean values of the sky brightness (in mag)  for the WFD survey and the Rolling
  Cadence. Median values of seasons corresponding to
  higher number of observations (i.e seasons (2,5,8), (3,6,9) and
  (4,7,10) for field types a, b, and c, respectively) have been
  considered for this estimation. }
\label{tab:rolling_sky}
\begin{tabular}{c|c|c|c|c}
\hline
\hline
   Band  & \multicolumn{2}{|c}{WFD}  &      \multicolumn{2}{|c}{Rolling} \\
            &     median & mean & median & mean \\
\hline
     u   &  22.0  & 22.05 $\pm$ 0.01  & 22.0  & 22.02 $\pm$ 0.01 \\
     g   &  21.8  & 21.78$\pm$ 0.01  & 21.8  & 21.80 $\pm$ 0.02  \\
     r   &   21.0  & 21.07 $\pm$ 0.01   & 21.1  & 21.16 $\pm$ 0.02  \\
     i   &   19.9  & 19.8 $\pm$ 0.2   & 20.1  & 20.10 $\pm$ 0.02 \\
     z  &   17.5  & 17.5 $\pm$ 0.1  & 19.3  & 19.30 $\pm$ 0.01 \\
    y  &    17.3  & 17.3 $\pm$ nan & 17.3   & 17.3 $\pm$ nan \\
\hline
\end{tabular}
\end{center}
\end{table}
Figure \ref{fig:m5_cadence_limits_rolling} shows the cadence {\em
  vs.}  $m_{5\sigma}$ plane for the wide survey.  The red stars
show the depth to attain, with a 3-day observer-frame cadence
(24.83, 24.35, 23.88 and 23.30, in $g, r, i$ and $z$ respectively). .  If we target a survey complete up to $z_{lim} \sim 0.4$
we find that:
\begin{itemize}
  \item in $r$ , and $ i$, we go deep enough with the standard visits;
  \item $g$ and $z$ bands are not deep enough. Reaching the goal
    $z_{lim} \sim 0.4$ would require to double the exposure time in
    the $z$ band;
  \item the $z_{lim} \sim 0.5$ seem reachable for the $r$ and $i$
    bands by tripling the exposure times, as it may be predicted from figure \ref{fig:m5_expoTime},
and as it is observed on \label{fig:m5_cadence_limits_rolling_triplexpo}.
\end{itemize}



\begin{figure*}[t]
\begin{center}
\subfigure[$g$]{\includegraphics[width=0.48\linewidth]{m5_cadence_limits_rolling_g.png}}
\subfigure[$r$]{\includegraphics[width=0.48\linewidth]{m5_cadence_limits_rolling_r.png}}\\
\subfigure[$i$]{\includegraphics[width=0.48\linewidth]{m5_cadence_limits_rolling_i.png}}
\subfigure[$z$]{\includegraphics[width=0.48\linewidth]{m5_cadence_limits_rolling_z.png}}
\caption{The cadence {\em vs.} depth requirements for the rolling cadence on the Wide survey,
  in the $g, r, i $ and $z$-bands. The color scale corresponds to the
  metric described in equation \ref{eqn:global_metric}.  The lines are
  the contour-levels computed for the limits indicated on table
  \ref{tab:cadence_depth_limit}. To fulfill the SNR requirements at a
  given redshift, one has to be {\em below} the corresponding
  line. The red and blue crosses show what the survey can currently
  deliver according to \code{Minion\_1016} and the Rolling cadence, respectively.}
\label{fig:m5_cadence_limits_rolling}
\end{center}
\end{figure*}

\begin{figure*}[t]
\begin{center}
\subfigure[$g$]{\includegraphics[width=0.48\linewidth]{m5_cadence_limits_rolling_g_3expTime.png}}
\subfigure[$r$]{\includegraphics[width=0.48\linewidth]{m5_cadence_limits_rolling_r_3expTime.png}}\\
\subfigure[$i$]{\includegraphics[width=0.48\linewidth]{m5_cadence_limits_rolling_i_3expTime.png}}
\subfigure[$z$]{\includegraphics[width=0.48\linewidth]{m5_cadence_limits_rolling_z_3expTime.png}}
\caption{The cadence {\em vs.} depth requirements for the rolling cadence on the Wide survey,
  in the $g, r, i $ and $z$-bands. The color scale corresponds to the
  metric described in equation \ref{eqn:global_metric}.  The lines are
  the contour-levels computed for the limits indicated on table
  \ref{tab:cadence_depth_limit}. To fulfill the SNR requirements at a
  given redshift, one has to be {\em below} the corresponding
  line. The red and blue crosses show what the survey can currently
  deliver according to \code{Minion\_1016} and the Rolling cadence
  with an exposure time of 90 seconds, respectively.}
\label{fig:m5_cadence_limits_rolling_triplexpo}
\end{center}
\end{figure*}

% ----------------------------------------------------------------------

\section{Conclusions}
\label{sec:conclusions}

We have presented two lightweight metrics to assess whether a given
cadence permits to build a redshift-limited sample up to a given
redshift $z_{lim}$.

We have presented design guidelines for the wide and deep LSST SN
surveys and discussed requirements on the survey depth and cadence.

We have evaluated the \code{Minion\_1016} cadence, on a series of DDF
and wide fields. On the DDF fields, we have found that the current
cadence yields a SN survey complete up to $z \sim 0.7$.  This is
comparable to what was obtained with SNLS in the 2000's but well below
what should be an ambitious goal for LSST.

We have shown that building a SN sample complete up to $z \sim 0.8$
from the DDF fields requires implementing a rolling cadence similar to
what has been proposed for the wide survey. This still represents a
very significant challenge, as it imposes increasing the exposure time
of the nominal DDF visits by about a factor 3.

On the wide survey, we have confirmed that the rolling cadence allows
one to reach our target depth ($z_{lim} \sim 0.4$). Reaching $z_{lim}
\sim 0.5$ is also within reach. It requires tripling the duration of
the $r$ and $i$-band visits.

The results presented here should be taken with a bit of caution.  As
we discuss in appendix \ref{sec:lsst_instrument_models}, and show on
figure \ref{fig:lsst_model_summary}, the assessment of the instrument
throughput and of the average observing conditions have been recently
revised, and this revision has a very significant impact on the
expected survey depth (almost half a magnitude per standard visit). We need to understand 
whether this revised model corresponds to some kind of ``worst case'' model, or wether 
it is indeed a realistic assessment of the instrument throughput and observing conditions.

Furthermore, the average observing conditions used in
\code{Minion\_{1016}} seem to be quite pessimistic.  In particular,
the \code{Minion\_{1016}} sky brightness is probably too bright,
especially in the $y$-band.

The results presented in this note will be updated as revised OpSim
cadences become available.


% ----------------------------------------------------------------------

\subsection*{Acknowledgments}

% Here is where you should add your specific acknowledgments, remembering that some standard thanks will be added via the \code{acknowledgments.tex} and \code{contributions.tex} files.

% \input{acknowledgments}

% \input{contributions}

% {\it Facilities:} \facility{LSST}

% Include both collaboration papers and external citations:
\bibliography{lsstdesc,main}





\appendix

\section{LSST instrument models}
\label{sec:lsst_instrument_models}

The measurement errors and 5-$\sigma$ depth used have been computed
either with two codes: (1) the
\href{https://www.lsst.org/scientists/simulations/maf}{Metric Analysis
  Framework} (MAF) used within LSST-DESC and (2) a recent rewrite of
{\tt snsim} supernova simulator that was used to carry out the
LSST-Euclid forecasts for presented in \cite{2014A&A...572A..80A}.
Both codes implement a very similar approach and give very similar
results, provided that they are fed with the same throughput curves,
PSF model, seeing and sky brightness values.

While comparing the two codes, we noticed that the official LSST
assessment of the telescope throughput has been revised very
significantly over the last few years. 



\begin{table}[t]
\begin{center}
\caption{LSE-40 model}
\label{tab:lse40}
\begin{tabular}{l|cccccc}
\hline 
\hline 
\multicolumn{7}{c}{{\bf General}} \\
\hline
Pixel size & \multicolumn{6}{r}{0.2 arcsec} \\
RO noise   & \multicolumn{6}{r}{9 $e^-$}    \\
\hline
\multicolumn{7}{c}{{\bf Zero Points @ X=1 [AB, fluxes in e$^-$/s]}} \\
\hline
           &  $u$ & $g$ & $r$ & $i$ & $z$ & $y$ \\
LSE-40     & 27.09 & 28.58 & 28.50 & 28.34 & 27.95 & 27.18 \\
snsim      & 27.05 & 28.59 & 28.53 & 28.38 & 27.99 & 27.22 \\
\hline
\multicolumn{7}{c}{{\bf median seeing [arcsec]}} \\
\hline
LSE-40 / snsim  &  0.77 &  0.73 &  0.70 &  0.67 &  0.65 &  0.63 \\
\hline
\multicolumn{7}{c}{{\bf Dark sky [AB mag / arcsec$^2$]}}   \\
\hline
LSE-40     & 22.92 & 22.27 & 21.20 & 20.47 & 19.59 & 18.63 \\
snsim      & 22.95 & 22.26 & 21.20 & 20.47 & 19.60 & 18.61 \\
\hline
\multicolumn{7}{c}{{\bf NEA [pixel$^2$]}}   \\
\hline
snsim (Moffat, $\beta=4.5$)     & 41.5  & 37.4  & 34.5  & 31.7 & 29.9  & 28.6  \\
\hline
\multicolumn{7}{c}{{\bf Limiting mag ($5 \sigma$), 30-s visit}}   \\
\hline
LSE-40                        & 24.22  &  25.15 &  24.74  &  24.38  &  23.80  &  22.93  \\
snsim (Moffat, $\beta=7$)     & 24.27  &  25.18 &  24.73  &  24.36  &  23.77  &  22.92  \\
\hline
\end{tabular}
\end{center}
\end{table}


{\tt snsim} was originally using numbers reported in the official,
Change Controlled Document \cite[][LSE-40 hereafter]{LSE-40}.  The
forecasts presented in \cite{2014A&A...572A..80A} were based on this
model.  The LSE-40 throughput curves are displayed on figure
\ref{fig:throughput_lse40}.  The key quantities of this model, in
particular, the expected median seeing and dark sky brightness are
listed in table \ref{tab:lse40}.  The connections between this model
of the observations and the LSST data products have been discussed in
depth in \citep{2008arXiv0805.2366I}.

The model presented in LSE-40 seems to have been revised recently,
although the LSE-40 document itself has not been changed (yet). The
update is described in \cite[][hereafter SMTN-002)]{SMTN-002}.  The
new throughput curves (extracted from MAF) are shown on figure
\ref{fig:throughput_smtn002}.  The key quantities (seeing, sky etc)
are listed in table \ref{tab:smtn002}.  It seems that \code{OpSim} and
\code{MAF} now rely on \code{SMTN-002}.  We have incorporated SMTN-002
into our \code{snsim} SN simulation code, and (unless specified
otherwise) all the computations presented in this code have been
carried out with SMTN-002.


\begin{table}[t]
\begin{center}
\caption{SMTN-002 model}
\label{tab:smtn002}
\begin{tabular}{l|cccccc}
\hline 
\hline 
\multicolumn{7}{c}{{\bf General}} \\
\hline
Pixel size & \multicolumn{6}{r}{0.2 arcsec} \\
RO noise   & \multicolumn{6}{r}{9 $e^-$}    \\
\hline
\multicolumn{7}{c}{{\bf Zero Points @ X=1 [AB, fluxes in e$^-$/s]}} \\
\hline
           &  $u$ & $g$ & $r$ & $i$ & $z$ & $y$ \\
SMTN-002   & 26.50 & 28.30 & 28.13 & 27.79 & 27.40 & 26.58 \\
snsim      & 26.48 & 28.34 & 28.17 & 27.85 & 27.46 & 26.63 \\
\hline
\multicolumn{7}{c}{{\bf median seeing [arcsec]}} \\
\hline
SMTN-002 / snsim  &  0.92 &  0.87 &  0.83 &  0.80 &  0.78 &  0.76 \\
\hline
\multicolumn{7}{c}{{\bf Dark sky [AB mag / arcsec$^2$]}}   \\
\hline
SMTN-002   & 22.95 & 22.24 & 21.20 & 20.47 & 19.60 & 18.63 \\ %% line 1 of Table 2
snsim      & 22.98 & 22.23 & 21.19 & 20.46 & 19.60 & 18.61 \\
\hline
\multicolumn{7}{c}{{\bf NEA [pixel$^2$]}}   \\
\hline
snsim (Moffat, $\beta=7$)     & 58.8  & 52.7  & 48.0  & 44.7  & 42.6  & 40.5  \\
\hline
\multicolumn{7}{c}{{\bf Limiting mag ($5 \sigma$), 30-s visit}}   \\
\hline
SMTN-002                    &  23.60     &  24.83     &  24.38     &   23.92    &  23.35     &  22.44  \\
snsim (Moffat, $\beta=7$)   &  23.61     &  24.83     &  24.35     &   23.88    &  23.30     &  22.43  \\
\hline
\end{tabular}
\end{center}
\end{table}


\begin{figure}[t]
\begin{center}
\subfigure[LSE-40]{\includegraphics[width=0.45\linewidth]{lse_40_passbands.pdf}\label{fig:throughput_lse40}}
\subfigure[SMTN-002]{\includegraphics[width=0.45\linewidth]{smtn002_passbands.pdf}\label{fig:throughput_smtn002}}
\caption{Instrument passbands}
\label{fig:throughput_comparison}
\end{center}
\end{figure}

We note howeverthat both models differ very significantly.  As can be
seen on figure \ref{fig:throughput_comparison} the throughput of
\code{SMTN-002} is almost 40\% lower than that of \code{LSE-40}, and
the associated zero points are lower by about 0.5-mag.  

\begin{figure}[t]
\begin{center}
\subfigure[Mirror reflectivities ($M_1 \times M_2 \times M_3)$]{\includegraphics[width=0.49\linewidth]{mirror_reflectivity.pdf}}
\subfigure[Optics transmission]{\includegraphics[width=0.49\linewidth]{optics_transmission.pdf}}
\subfigure[CCD QE]{\includegraphics[width=0.49\linewidth]{ccd_qe.pdf}}
\subfigure[Filters]{\includegraphics[width=0.49\linewidth]{filters.pdf}}
\caption{Systematic comparison of the constituents of the model
  passbands. For the CCD QE, we over plot the average of recent QE
  curves obtained by the project, for E2V and ITL sensor candidates. }
\label{fig:passband_constituents}
\end{center}
\end{figure}

On figure \ref{fig:passband_constituents} we compare the ingredients
of the LSE-40 and SMTN-002 throughput curves.  We see that the main
difference come from the mirror reflectivity. The overall transmission
of the optics follow, as well as the assessment of the filter peak
transmission.  It would be very useful to know whether SMTN-002 is
considered a non-consevative assessment of what is expected or whether
it is some kind of worst case model.

The the median seeing values have been revised upwards very
significantly. We do not know yet whether it comes from a revision of
the site image quality, or whether is is related to changes of the PSF
model. In any case, we are able to reproduce the 5-$\sigma$ depths
quoted in \cite{LSE-40} and \cite{SMTN-002} with \code{snsim} without
changing the PSF model (we use a moffat PSF with $\beta=7$).

Regarding the dark sky brightness, we note that it has also been
revised upwards quite significantly (see tables \ref{tab:lse40} and
\ref{tab:smtn002}).

All these changes, combined, make that the 5-$\sigma$ depth of the
standard 30-s LSST visist has to be revised downwards by about
0.5-mag. This is a significant downgrade, which impacts severely the
cost of the LSST SN survey (the DDF survey in particular).

Finally, we note that \code{OpSim} attempts to predict realistic
values of seeing and sky brightness (LSE-40 and SMTN-002) list values
for observations at zenith and dark sky. The median values obtained
from \code{OpSim} are significantly higher. The resulting 5-$\sigma$
depth is almost 0.5-mag and 1-mag lower than what can be inferred from
SMTN-002 and LSE-40 respectively. 

\begin{figure*}[t]
\begin{center}
\subfigure[$u$]{\includegraphics[width=0.48\linewidth]{m5_vs_expoTime_u.png}}
\subfigure[$g$]{\includegraphics[width=0.48\linewidth]{m5_vs_expoTime_g.png}}\\
\subfigure[$r$]{\includegraphics[width=0.48\linewidth]{m5_vs_expoTime_r.png}}
\subfigure[$i$]{\includegraphics[width=0.48\linewidth]{m5_vs_expoTime_i.png}}\\
\subfigure[$z$]{\includegraphics[width=0.48\linewidth]{m5_vs_expoTime_z.png}}
\subfigure[$y$]{\includegraphics[width=0.48\linewidth]{m5_vs_expoTime_y.png}}
\caption{$m_{5}-m_{5}^{30}$ as a function of the exposure time,
  normalized to 30 seconds. $m_{5}$ is the (5$\sigma$) depth for a given
  exposure time, and $m_{5}^{30}$ is the (5$\sigma$) depth
  corresponding to an exposure time of 30 seconds. An airmass value of
1.2 was used.}
\label{fig:m5_expoTime}
\end{center}
\end{figure*}

It may be interesting to estimate $m5\sigma$ as a function
of exposure time so as to predict the observation time required to
reach a given depth. Figure \ref{fig:m5_expoTime} tend to show that an
increase of 0.5 (1) magnitude in $m5\sigma$ requires to increase
two-to-threefold (sixfold)  the default exposure time of 30 second.

\section{Additional figures}

\begin{figure}[t]
  \begin{center}
    \includegraphics[width=\linewidth]{metric_DD_744.pdf}
    \caption{}
  \end{center}
\end{figure}


\begin{figure}[t]
  \begin{center}
    \includegraphics[width=\linewidth]{metric_DD_1427.pdf}
    \caption{}
  \end{center}
\end{figure}

\begin{figure}[t]
  \begin{center}
    \includegraphics[width=\linewidth]{metric_DD_2412.pdf}
    \caption{}
  \end{center}
\end{figure}

\begin{figure}[t]
  \begin{center}
    \includegraphics[width=\linewidth]{metric_DD_2786.pdf}
    \caption{}
  \end{center}
\end{figure}



\begin{figure*}[t]
  \begin{center}
    \includegraphics[width=\linewidth]{metric_WFD_660.pdf}
    \caption{}
  \end{center}
\end{figure*}

\begin{figure*}[t]
  \begin{center}
    \includegraphics[width=\linewidth]{metric_WFD_1088.pdf}
    \caption{}
  \end{center}
\end{figure*}




\end{document}
% ======================================================================
% 








% There are a number of useful \LaTeX\xspace commands predefined in
% \code{macros.tex}.  Notice that the section labels are prefixed with
% \code{sec:} to allow the use of the \verb=\secref= command to
% reference a section (\ie, \secref{intro}).  Figures can be referenced
% with the \verb=\figref= command, which assumes that the figure label
% is prefixed with \code{fig:}.  In \figref{example} we show an example
% figure.  You'll notice that the actual figure file is found in the
% \code{figures} directory.  However, because we have specified this
% directory in our \verb=\graphicspath= we do not need to explicitly
% specify the path to the image.

% The \code{macros.tex} package also contains some conventional
% scientific units like \angstrom, \GeV, \Msun, etc. and some editorial
% tools for highlighting \FIXME{issues}, \CHECK{text to be checked},
% \COMMENT{comments}, and \NEW{new additions}.
